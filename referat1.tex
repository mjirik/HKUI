\documentclass[12pt]{article} % velikost písma a typ dokumentu
\usepackage[utf8]{inputenc}  % kódování
\usepackage[czech]{babel}  % jazykový balík
\usepackage{graphicx}  % balik nutný pro vkládání obrázků
\usepackage{indentfirst}  % odsazení prvního řádku v odstavci (někdy se to dělá, někdy ne - co je správně nevím)

\begin{document}  % začátek dokumentu - na konci je end ;)

\begin{figure}[h!] % vkládání plovoucího prostředí pro obrázky parametr, vynucuje umístění here!
\centering % zajištění centrování - pro aktuální 'figure'
\includegraphics[width=0.4\textwidth]{pic/FAV_cmyk.png} % vložení obrázku - i s cestou jeho umístění !
\hspace{2cm}  % vložení horizontální mezery
\includegraphics[width=0.4\textwidth]{pic/ZCU_cmyk.png} % vložení obrázku
\end{figure}

\begin{center} % centrování textu (begin - end)
\LARGE{Sauron Gorthaur}\\ % lokální modifikace velikosti textku (relativně k definici v documentclass)
\vspace{.5cm} % vložení vertikální mezery
\end{center}

\begin{figure}[h!]
    \centering
    \includegraphics{pic/sauron-foto.jpg}  
    \caption{\textit{"Ash nazg durbatulûk, Ash nazg gimbatul, Ash nazg thrakatulûk, agh burzum-ishi krimpatul."} - Sauron}
                                            % mezera mezi řádky zajišťuje zalomení řádku !
    % \large{}  % modifikace textu velikost, italica
\end{figure}

\vfill{}  % vyplnění 'prázdnem' - zajistí, že zbytek bude dole na stránce 
\noindent  % zabrání odsazení nového odsatvce
Západočeská Univerzita V Plzni \hfill mr. Example\\  % vyplnění prázdnem - horizontální
Katedra Kybernetiky \hfill 31. semestr\\
Semestrální práce - HKUI \hfill \today % today vypíše datum - ve formátu definovaném volbou jazyka. 
\thispagestyle{empty}  % zakáže číslování stránky
\newpage % zalomení stránky
\setcounter{page}{1} % nastaví počítadlo stránek

\section{Životopis} 
\subsection{Počátek}
Sauron patřil k Aulëho lidu, projevoval se jako velký řemeslník. Sauron byl původem mnohem vyššího řádu než např. Gandalf nebo Saruman, a největší, nejnebezpečnější, a nejspolehlivější Melkorův služebník, jeho pobočník v Angbandu, a po jeho pádu sám Temný pán a Nepřítel.

Eru ponechal andělské duchy hrát Hudbu Ainur (Ainulindalë), kteří rozvíjeli téma zjevené samotným Eru. Melkor se však pokoušel zvýšit svou slávu ve své písni svými myšlenkami a ideami, které nebyly v souladu s původním Tématem Hudby. „… a rázem kolem něho vznikl nelad a mnozí, kteří zpívali blízko něho, propadli sklíčenosti, jejich myšlenky se narušily a jejich hudba zakolísala; někteří však začali ladit svou hudbu spíš podle něho než podle myšlenky, kterou měli zprvu.“

Melkorova disharmonie měla své strašlivé důsledky, jelikož Eru dovolil Ainur vykonat „Píseň Stvoření“, jako šablonu světa, který byl stvořen: „Zlo světa nebylo na poprvé ve velkém Tématu, ale vstoupilo s disharmonií Melkora.“ Sauron nebyl původcem disharmonie a nepochybně věděl víc o Hudbě než Melkor, jehož mysl byla vždy naplněna jeho vlastními plány a záměry. Zřejmě Sauron nebyl jedním z duchů, kteří okamžitě ladili svou Hudbu podle Melkora.

Brzy Melkorova disharmonie jako by byla ve válce s Tématem Eru - Hudba reprezentuje konflikt dobra a zla. Nakonec Eru píseň ukončil. Aby duchové viděli, co učinili, Eru učinil bytí Hudby. Tak byl stvořen vesmír Eä, ve kterém měl být konflikt dobra a zla rozhodnut. Eru umožnil duchům vstoupit do Eä. Mnoho tak učinilo a Sauron byl mezi nimi. Tím, že jim umožnil vstoupit do světa, umožnil jak velké zlo, tak velké dobro. 

\subsection{První věk}
Po vstupu do Eä na začátku času, se Valar a jejich Maiar pokusili vybudovat svět podle vůle Ilúvatara. Ve svém úsilí se Sauron projevil jako „velký řemeslník patřící k Aulëmu“. Jako ostatní Valar, Aulë učil své služebníky Maiar o struktuře, zákonech a podstatě světa a Sauron získal tuto znalost.

Uvnitř obrovských prostor Eä Valar zaměřili své úsilí na Ardu, Zemi, kde elfové a lidé budou žít. Ale Melkor, později známý jako Morgoth, také přišel do Ardy. Toužíc stát se svrchovaným pánem světa oponoval ostatním Valar, kteří zůstali věrní Ilúvatarovi a snažili se sledovat Stvořitelův plán. V této době Sauron podlehl Melkorově vlivu. Tolkien poznamenal, že Sauron obdivoval pořádek a koordinaci a nesnášel zmatek a neuspořádané třenice. Tak moc a vůle Melkora po nastolení pořádku v Ardě Saurona přitahovala.

Sauron měl podíl na všech činech Melkora na Ardě a byl jen o to méně zlý, že zpočátku sloužil jinému a ne sobě. Po jistou dobu zřejmě předstíral, že je věrným služebníkem Valar, zatímco předával Melkorovi informace o všem, co činili. Když Valar vytvořili říši Almaren, osvětlenou svitem Dvou lamp, Melkor o všem věděl, protože měl zvědy mezi Maiar a Sauron byl jejich náčelníkem.

Melkor zničil Almaren a Valar vytvořili novou zemi na Nejzazším Západě: říši Valinor. Valar ještě nevěděli o Sauronově zradě, a tak Sauron zůstal mezi nimi. Teprve později Sauron Valinor opustil a odešel do Středozemě, centrálního kontinentu Ardy. Od té chvíle byl Sauron definitivně na straně Melkora. Poté, co se přidal k Melkorovi ve Středozemi, prokázal, že je oddaným a schopným služebníkem, a Melkor mu svěřil správu své pevnosti Angband.

Ta nebyla po Druhé válce s Melkorem (brzy po procitnutí elfů) příliš prozkoumávána, takže Sauron nebyl dopaden. Po Melkorově návratu se stal jeho nejmocnějším sluhou, byl pánem mučení. Po bitvě Náhlého plamene ovládl Tol Sirion, kam se nastěhoval se svými vlkodlaky a kde věznil mimo jiné Finroda Felegunda a Berena, než byl nucen předat vládu nad věží Lúthien Tinúviel. Po Válce hněvu se Sauron zatoužil napravit a prosil Eönwëho, herolda Valar, o odpuštění, ten však neměl pravomoc odpustit bytosti vlastního řádu, a proto jej vyzval, aby se s ním vrátil do Amanu, kde předstoupí před Soudný kruh Valar. Sauron se však zalekl přísného trestu a raději zůstal ve Středozemi, kde se ukryl daleko na Východě. 

\end{document}


